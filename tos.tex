We, the Student Robotics contributors ('the blueshirts') organize and
operate the Student Robotics Competition ('the competition'), an
engineering challenge for 16-18 year olds. We provide an electronics kit
('the kit') on loan, software tools, documentation and support to help
the competitors build a robot that performs a task -- free of charge.

The robot task ('the game') is set at the beginning of the academic year
at the Kickstart event, after which the competitors have several months
to build their robots, in schools at at home. 'Tech day' support
sessions are available at various locations during the build period. At
the end of this period a competition is held to test how their robots
perform. These events will all announced on our website (see the 'key
dates' section) and operated by our contributors [[1]]. We will also
provide our own risk assessments for the events in advance [[2]].

Our primary means of support is via the website, which contains
documentation for all our equipment and our programming environment. The
website also has a forum for competitors to ask questions and discuss
ideas, and the IDE, a web page for writing code for the robot. To access
some of these resources, competitors will be provided with personal
accounts. To maintain these accounts, we store the following pieces of data:
 * First name,
 * Last name,
 * Contact email address,
 * Which team/school the competitors is part of.

We will not share any of this information with third parties, except
when printing participation certificates, when we will send a list of
names to our printers. The email address will only be used to send
(infrequent) bulk organization email to students, although in
exceptional circumstances we may email the student directly regarding
something they've done on the website.

[Do we want to say anything about mentoring?]

~

To make the whole competition happen, we also need you, the 'team
leader', to take responsibility for a number of things. Foremost of
these is the care of the competitors: while we provide a safe [[1]]
environment at all our events, we are not able to provide parental
responsibility. It is up to you to get parental permission and
arrange/provide suitable supervision at our events. For this reason, we
require you to not be competing in the competition yourself [[3]].

We are also unable to provide transportation, meals or accommodation  at
any of the events. (Note that the competition is 2 days long).

You are also responsible for ensuring competitor safety during the robot
build period, such as the safe use of tools. The electronics kit we
provide is mostly safe, however the batteries provided can be dangerous
if mistreated. We will provide a risk assessment [[2]] explaining this,
along with battery charging instructions [[4]].

Please also ensure that the kit is treated carefully -- while we expect
some breakage to occur over the competition period, and can provide
spares, we would prefer this to be kept to a minimum.

Many teams wish to keep their kit after the competition event, to
demonstrate their robot or otherwise work with it. Please let us know
before the competition if the team wish to keep it for a longer period,
and we can arrange a mutually agreeable return date. Unfortunately, we
are not in a position to let anyone keep the kit permanently, or sell
it. If you keep the kit after the competition, you will need to return
the kit at your own expense -- if posted, please ensure it is registered
delivery, with at least £500 worth of insurance.

We will also need you to supply the competitor account data (described
above) to an online web form, so that we can create competitors
accounts. Please ensure that competitors use their own account when
accessing the website: in the (extremely) rare case of an account being
abused we will disable the account, and we'd rather not lock the whole
team out in the process.
